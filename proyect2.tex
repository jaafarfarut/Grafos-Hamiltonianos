\documentclass[11pt,a4paper]{article}
\usepackage[utf8]{inputenc}
\usepackage{amsmath}
\usepackage{amsfonts}
\usepackage{amssymb}
\usepackage{multicol}
\usepackage{graphicx}
\usepackage{multicol}
\newcommand{\ds}{\displaystyle}
\pagestyle{empty}
\usepackage[left=2.00cm, right=2.00cm, top=2.00cm, bottom=2.00cm]{geometry}
\usepackage[usenames]{color}
\begin{document}
\begin{center}
\LARGE \noindent UNIVERSIDAD NACIONAL DE INGENIERÍA\\
Facultad de Ciencias\\
Matemática y Ciencia de La computaci{\'o}n
\end{center}
\begin{center}
\begin{minipage}[c]{9cm}
\includegraphics[scale=1.00]{uni.PNG}
\end{minipage}
\end{center}
\begin{center}
\Large \textbf{T{\'I}TULO DEL TRABAJO}\\
Verificaci{\'o}n de la existencia de un ciclo hamiltoniano en un grafo aleatorio
\end{center}
\begin{center}
\Large \textbf{Unidad Acad{\'e}mica}: 
\\Facultad de Ciencias\\
\textbf{Curso y secci{\'o}n}: 
\\Introducci{\'o}n a la Estad{\'i}stica\\ y Probabilidades(CM-274 "A")\\
\textbf{Semestre}: 
\\2018-II\\
\textbf{Profesores}:
\\ Zamudio Fernando - C{\'e}sar Lara\\
\textbf{Integrantes}:\\
/Jaafar Farut Sahua Torres/\\
/Franklin F{\'e}lix Rivera Granados/\\
/Briguitte Stefany Maquera de la Cruz/\\
\end{center}
\begin{center}
\LARGE \vfill\textbf{Lima-Per{\'u}}\\
\textbf{(2018)}
\end{center}

\begin{center}
\LARGE \textbf{\textcolor{blue}{Introducci{\'o}n}}\\[1cm]
\end{center}
\Large
\begin{multicols}{2}
\textbf{¿\underline{Qu{\'e} es un Grafo}?}\\
• \textbf{Grafo}: Es un diagrama que representa mediante vertices y aristas las relaciones entre pares de elementos y que se usa para resolver problemas l{\'o}gicos, topol{\'o}gicos y de c{\'a}lculo combinatorio.\\
• \textbf{Grafo hamiltoniano}: Es aquel grafo que tiene un ciclo hamiltoniano el cual recorre una sola vez cada vertice y el vertice final sea adyacente al primero, de esa forma contiene un camino hamiltoniano o circuito hamiltoniano.\\
\textbf{¿\underline{C{\'o}mo identificar un}}\\
\textbf{\underline{grafo hamiltoniano}?}\\
Contrario al caso de los grafos eulerianos, para el caso de los grafos hamiltonianos no se conoce ninguna condici{\'o}n necesaria y suficiente que los caracterice. Esto es lamentable porque en muchas aplicaciones es fundamental poder determinar si un grafo es hamiltoniano.\\
\textit{\textbf{Ejemplos de Grafos hamiltonianos}}\\
\begin{minipage}[c]{4cm}
\includegraphics[scale=0.42]{grafo.PNG}
\end{minipage}\\[1cm]
\textbf{¿\underline{Qu{\'e} es el Lenguaje}}\\
\textbf{\underline{de programaci{\'o}n R}?}\\
Es un tipo de lenguaje de programaci{\'o}n el cual es una implementaci{\'o}n del lenguaje de programaci{\'o}n S, creado en Auckland(New Zealand)\\
•\textbf{Caracter{\'i}sticas}\\
\textbf{*} R es un lenguaje pensado para la programaci{\'o}n estad{\'i}stica y la creaci{\'o}n de gr{\'a}ficos\\
\textbf{*} Posee mucho paquetes y librerias\\
\textbf{*} Es multi-paradigm{\'a}tico y Open Source ya que nos permite una facilidad en el uso de la escritura o implementaci{\'o}n del c{\'o}digo\\
\textit{\textbf{Nota:}}RStudio es un entorno de desarrollo integrado (IDE) para el lenguaje de programación R, dedicado a la computación estadística y gráficos.\\
\begin{minipage}[c]{4cm}
\includegraphics[scale=0.41]{R.PNG}
\end{minipage}
\end{multicols}

\begin{center}
\LARGE \textbf{\textcolor{blue}{Objetivo del Proyecto}}\\[1cm]
\end{center}
\Large
\begin{multicols}{2}
• Es la verificaci{\'o}n de un grafo y determinar si es o no es hamiltoniano  pues ya que aunque no hay condici{\'o}n o formula totalmente eficiente para su demostracion, podemos aproximarnos utilizando ciertas condiciones.\\
• El implemento de la programacion mediante el uso del Lenguaje R en nuestro proyecto para dicha verificacion\\
• El uso de algunas formulas y teoremas estadisticos para la determinaci{\'o}n de un grafo y verificar si es o no es hamiltoniano\\[5cm]
\end{multicols}

\begin{center}
\LARGE \textbf{\textcolor{blue}{Estado del arte}}\\[1cm]
\end{center}
\Large
\begin{enumerate}
\begin{multicols}{2}
El problema respecto a los grafos, espec{\'i}ficamente los ciclos hamiltonianos, vienen siendo una adversidad hist{\'o}rica.\\
En \textbf{1736} Leonhard Euler en uno de sus viajes en Koñigsberg en la costa del Mar B{\'a}ltico, en la Prusia oriental (Rusia) hab{\'i}an siete puentes distribuidos donde plane{\'o} un paseo de manera que saliendo de casa cruce los siete puentes una sola vez cada uno antes de regresar a casa haciendo referencia a los caminos hamiltonianos.\\
En \textbf{1805} Roman Hamilton se propuso a viajar a 20 ciudades del mundo, representadas como los v{\'e}rtices de un dodecaedro regular, siguiendo las aristas del dodecaedro.\\
En \textbf{1824} Kirchoff se sirvi{\'o} de la Teor{\'i}a de Grafos para enunciar las leyes que permiten el c{\'a}lculo de voltajes y circuitos el{\'e}ctricos.
\end{multicols}
\end{enumerate}

\begin{center}
\textbf{\textcolor{blue}{Diseño del experimento}}\\
\end{center}
\end{document}
